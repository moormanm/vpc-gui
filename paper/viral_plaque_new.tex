
\documentclass[11pt,final,twocolumn]{IEEEtran}
\usepackage{graphicx}
\usepackage{expdlist}
\usepackage{fancyvrb}
\usepackage{float}

\bibliographystyle{IEEEtran}



\renewcommand{\figurename}{Figure}
\renewcommand{\thefigure}{\Roman{figure}}
\restylefloat{pc}
\floatname{pc}{Figure}
\newfloat{pc}{H}{lop}
\setlength{\parindent}{0pt}
\setlength{\parskip}{1ex}
\begin{document}

\title{Automatic Viral Plaque Counting Using Image  Analysis Techniques}
\author{Michael Moorman, Aijuan Dong \\
Hood College}
\maketitle
\section{Abstract}
The manual counting of viral plaques is a tedious and laborious process. In this paper, an efficient and economical method is proposed for automating viral plaque counting via image segmentation and various morphological operations. The method first segments the whole image into individual plate images. Then, it converts each plate image into binary image and creates a new image by merging the dilated binary image and the complement image of the eroded binary image. At last, the contour hierarchy of the merged image is obtained and the plaque count is calculated by evaluating each outer contour count and its inner contour counts. Experiment results showed that the counting accuracy for the proposed method is up to 90 percent and the average processing time for a single image is about one second. An open source implementation with optional graphical user interface is available for public use.


\section{Introduction}
The viral plaque assay is an important and commonly used procedure in biology; however, a stage of the process requires that a person manually count the instances of plaque structures . The work presented here attempts to accurately count mammalian  viral plaques using image processing techniques. The image below is a typical example of an input image.
\begin{figure}[h]
\centering
\includegraphics[width=.4\textwidth]{sample.jpg}
\caption{A typical example of a viral plaque image}
\label{fig:sample}
\end{figure}


A common example of the procedure begins by preparing several dilutions of a virus stock. Tissue plates, arranged in a batch of 6x4 wells, are configured with a thin monolayer of susceptible animal tissue sample.  The wells are then innoculated using the various virus dilutions. After inncoulation, the wells are topped off with a special agar that limits viral expansion to only neighboring cells. Each sample is allowed to incubate for some amount of time, allowing the virus to attack the cells In the monolayer. Over time, the virus gradually destroys neighboring cells, creating visible structures also known as viral plaques. The visibility of these plaques are sometimes enhanced with the usage of special dyes. Once the plaques have grown large enough to be seen by the naked eye, the titer of the original virus stock is discovered by observing the number of  plaque-forming units (PFU) per milliliter present in each well.

The last stage of this procedure can be particularly laborious. Counts of the number of plaque formations on each well are manually tallied by hand. It is also a subjective exercise, with different technicians frequently reporting different counts for a given sample.


Automated counting of viral plaques via image processing techniques  has been proposed and investigated many times before, and there are several commercial products on the market that manufacture colony counting systems. There is a lucrative market for automated colony counting, and several companies manufacture colony counting systems. These systems can cost thousands of dollars and are often both closed source and not economically feasible. Some examples include the Revolutionary Science IncuCount Automatic Colony Counter priced at \$$7,845$ ~\cite{labSafetySupply}, the UVP Imaging Sys EC3 410 priced at \$$31,179$ \cite{labPlanet}, and the Flash and Grow Automatic Colony Counter priced at \$$9,980$ \cite{coleParmer}. 

 Other studies have approached automated viral plaque counting before and have tried a number of different  approaches. In ~\cite{watershed}, a watershed transform was used to analyze a single dish image acquired by an overhead camera. This system was capable of counting to within $93$\% of
manual counting, however mammalian type plaque structures were not tested, and the implementation required up to ten seconds to process a single plate. A similar approach was used in ~\cite{distanceTransform}, where a distance trasform was used instead of watershed. It includes a software suite that was capable of counting an entire plate of samples to within $95$\% of human counts, but the process requires users to manually identify each area of interest on the plate using a graphical user interface.  Another approach used in ~\cite{houghTransform} used images of stained culture flasks and a Hough transform technique that produced counts falling well within the
distribution of corresponding manual counts, but it required a special image acquisition apparatus and did not allow for batch processing of an entire plate. 

At present time,  there is not a low cost and  accurate colony counting system available that enjoys wide spread adoption and that is able to accommodate an entire plate of samples. This is a primary motivation for the work presented here. In addition, the performance of some of these works was poor. 

Accurately counting plaques in these types of images presents several challenges. Perhaps the most difficult challenge that any solution must address is how to deal with the noise in the image.  How does one go about discerning the difference between viral plaques and the frequent noisy particles in the agar of the plates?  This is particularly problematic when dealing with small plaques, as there signal to noise ratio is very small in that case.  

OpenCV is an open source computer vision library available from http://sourceforge.net/projects/opencvlibrary. The library is written in C and C++ and runs under Linux, Windows, and Mac OS X. It provides an application interface (API) that this study uses to implement the proposed method. 


The rest of the paper is organized as follows. In Section \ref{sec:methodology}, a technical walk through of the counting method is presented. Experiment results are shown in Section \ref{sec:experiments} and the paper concludes in Sections \ref{sec:discussion} and \ref{sec:conclusion}.


\section{Methodology}\label{sec:methodology}
The proposed method is implemented as a computer program written in C++ and uses the  OpenCV ~\cite{openCV} image processing API to perform image analysis. 

\begin{figure}[H]
\centering
\includegraphics[width=.45\textwidth]{flowchart.png}
\caption{Overview of technical approach}
\label{fig:flowchart}
\end{figure}

\subsection{User Supplied Parameters}
The user supplies the program with an image of a plate of viral plaques, and also specifies three parameters.
\subsubsection{Well radius}
The radius of each well, given in pixels.

\subsubsection{Minimum Plaque Radius}
 The minimum radius of a viral plaque structure, given in pixels.

\subsubsection{Maximum Plaque Radius}
 The maximum radius of a viral plaque structure, given in pixels.


The method uses these parameters and a series of procedural morphological image transformations ~\cite{morphological} to ultimately produce viral plaque counts.

\subsection{Image Aquisition}
Image acquisition  is the first stage of the method. A flat bed scanner is used to capture an image of a plate of viral plaque samples, and they are encoded into an image file. Once acquired, the colors of the image are inverted. Color inversion is not a technical neccessity for the method, but it makes viral plaques easier to discern for the human eye.

\subsection{Segmentation}
Segmentation in the next stage of the method. The only interesting parts of the image lay within each well. Segmentation is an essential step that attempts to create a mask that shadows every part of the image that is not with in the barriers of a well on the plate. 

Perhaps the most difficult problem that segmentation seeks to resolve is the plate orientation problem. Image acquisition makes no guarantees as to the position of the plate with relation to the scanner surface. Plates may be butted up against the top left corner of the scanner, have a slight offset on any axis, or have any slight orientation about them, etc. A global segmentation approach overcomes these challenges. This approach is summarized below.

\begin{enumerate}
\item Grayscale the image.
\item Use Otsu's thresholding method ~\cite{otsu} to transform the image into a binary image. This is depicted in figure \ref{fig:segOtsu}.
\begin{figure}[H]
\centering
\includegraphics[width=.4\textwidth]{segmentOtsu.jpg}
\caption{Otsu thresholding}
\label{fig:segOtsu}
\end{figure}


\item
Perform a morphological dilation on the image that is sufficient enough to close viral plaque holes. This is shown in figure \ref{fig:segFillHoles}.
\begin{figure}[H]
\centering
\includegraphics[width=.4\textwidth]{segmentFillHoles.jpg}
\caption{Use of dilation to fill holes.}
\label{fig:segFillHoles}
\end{figure}



\item
Aggressively perform several iterations of morphological erosion on the image until the number of artifacts that remain on the image equal the number of wells on the plate. This is shown in figure \ref{fig:segErode}.
\begin{figure}[H]
\centering
\includegraphics[width=.4\textwidth]{segmentErode.jpg}
\caption{Use of massive erosion}
\label{fig:segErode}
\end{figure}


\item
Calculate the center points of each remaining artifact. Use these points as seeds for a floodfill ~\cite{floodfill} operation that grows the region outward from each of these points on an RGB thresholded rendering of the original image.
Aggressively perform several iterations of morphological erosion on the image until the number of artifacts that remain on the image equal the number of wells on the plate. This is shown in figure \ref{fig:segFloodFill}.
\begin{figure}[H]
\centering
\includegraphics[width=.4\textwidth]{segmentFloodFill.jpg}
\caption{Use of flood fill}
\label{fig:segFloodFill}
\end{figure}

\item
Recalculate the centroids of the new artifacts. Draw a circle at these origins of radius specified by the well radius parameter. This is shown in figure \ref{fig:segUseRadius}.
\begin{figure}[H]
\centering
\includegraphics[width=.4\textwidth]{segmentUseRadius.jpg}
\caption{Use of well radius parameter}
\label{fig:segUseRadius}
\end{figure}

\end{enumerate}

Segmentation produces clean regions of interest for each well in the plate. This part of the method is very effective at identifying regions of interest. It exploits the observation that images of plates used for viral plaque studies tend have a similar overall shape. The method expects that the image artifacts that represent the agar or agar equivelant in the wells are the most predominant features of the overall image.    Thus, the image can be reliably eroded until only these features remain.  

Segmentation also exploits the fact that the wells of the tissue plates are machined to be perfectly round. This knowledge combined with the well radius parameter that the user provides allows this phase of the method to produce clean regions of interest, which is vastly important to the success of later processing stages.

\subsection{Plaque Structure Counting}
Next remains the task of counting each viral plaque in each well. Given the often noisy nature of the substrate and viral plaques in the well, it’s difficult to devise a method that is accurate for all varieties of input. Viral plaques can be very small, very large, malformed, or even overlapped on top of one other. Nevertheless, the plaque counting method attempts to be simple and accurate for the most commonly occurring instances of viral plaques. It is summarized below.  

\begin{enumerate}
\item Let a masked region of interest be image A. This is a color image containing a single well.
\begin{figure}[H]
\centering
\includegraphics[width=.25\textwidth]{countBegin.jpg}
\caption{Image of a single well, with uninteresting regions masked out}
\label{fig:countBegin}
\end{figure}


\item Grayscale and then apply Otsu thresholding upon image A.
\begin{figure}[H]
\centering
\includegraphics[width=.25\textwidth]{countOtsuA.jpg}
\caption{Otsu thresholding applied to image A}
\label{fig:countOtsuA}
\end{figure}

\item Let images B,C, D, and E be zeroed copies of image A.
\item On image B, render an eroded image.
\begin{enumerate}
\item
Render a morphological open \cite{morphological} operation on image B using a 2x2  rectangular shaped kernel. Choose the number of iterations based on a constant derived from the maximum plaque radius parameter.
\begin{figure}[H]
\centering
\includegraphics[width=.25\textwidth]{countErodeB.jpg}
\caption{Morphologically eroded rendering of image B.}
\label{fig:countErodeB}
\end{figure}
\end{enumerate}

\item On Image C, render a dilated image.
\begin{enumerate}
\item  Find every contour in A. If the contour area is smaller than a constant derived from the minimum plaque area, remove it from the image.
\item  Dilate C using a 2x2 rectangular shaped kernel. Choose the number of iterations based on a constant derived from the minimum plaque radius parameter.
\item  Perform an  operation similar to the MATLAB imfill() ~\cite{matlab}  on image A to close any internal holes in the contours.
\begin{figure}[H]
\centering
\includegraphics[width=.25\textwidth]{countDilateC.jpg}
\caption{Morphologically dilated rendering of image C.}
\label{fig:countDilateC}
\end{figure}
\end{enumerate}

\item
Merge images B and C into a new image A using a binary image merge of the form 
\begin{Verbatim}[samepage=true]
A = ~B & C
\end{Verbatim}
\begin{figure}[H]
\centering
\includegraphics[width=.25\textwidth]{countMerged.jpg}
\caption{Images B and C merged together}
\label{fig:countMerged}
\end{figure}

\item
Obtain the hierarchy of contours in A. Let the total number of contours be determined by the following pseudocode:
\begin{Verbatim}[samepage=true]

minP = min plaque area
maxP = max plaque area
for contour C in outerContours:
  if( area(C) < minP):
    continue;

  if( area(C) > maxP:
    totalContours += 1;

  totalContours +=
     max(1, innerContourCount) 
\end{Verbatim}

\end{enumerate}
The plaque counting method attempts to find a balance between finding feint and sparse plaques VS. properly dividing and detecting dark, prominent plaques. Feint  plaques are typically discovered through analysis of the dilated image (Image C). Dilation effectively joins plaque clusters and forms a cohesive, filled areas. The erosion rendition of the well is focused on discerning prominent plaque structures.  Without an erosion step, a cluster of viral plaque structures are at risk of being inaccurately tallied as a single plaque. Combining the dilated and eroded versions of a well image takes advantages of both morphological transforma ions. Small, feint plaques are emphasized and large  clustered plaques are separated. Both transformations scale with the user supplied parameters maxPlaqueRadius and minPlaqueRadius.

The final step of the per well counting routine considers the max and minPlaqueRadius parameters when tallying total plaque counts. 

\section{Experiments}\label{sec:experiments}

For these experiments, viral plaque images are logically grouped as a study. In a given study, plates are prepared with similar parameter and all of the viral plaque plates in a study have been allowed to incubate for about the same amount of time. Each study consists of plates that use the same or very similar innoculation techniques. They also use similar agar, tissue mediums, and viruses for experiments. This organization scheme allows for a pairing of one set of program parameters to be attached to all the images for a given study. Likewise, parameters are configured for the program on a per study 
basis.

Truth data for each viral plaque well in each plate was obtained by manually counts. These results were compared with data produced by the computer program. Accuracy was measured by finding the sum of errant counts for each well and then dividing that sum by the total truth data count for the well. Errant counts for each well were determined by finding the absolute difference between a count produced by the program and the truth data for a given well.

Each image was aqcuired using an Epson Perfection 799 flat bed scanner, configured to capture at 24 bit color mode and 600 dots per inch. In each experiment, the wells were prepared using a $.5$\%   Methyl Cellulose overlay in lieu of agar. A $.4$\% solution of Crystal Violet was used as a dye as well. 
The plates used in each study were Costar/Corning 24 well (6x4) Tissue culture plates. 

\subsection{Study One}
In the first study, African Green Monkey Kidney Cells  Cell strain Vero E6 were infected with a Vaccinia Virus Western Reserve strain. Seven sample images were captured and sent to the program for processing. Each plate contained 24 wells, and a total of 168 data points were compared with corresponding truth data points.

\begin{figure}[H]
\centering
\includegraphics[width=.5\textwidth]{Study1Results.png}
\caption{Study one results}
\label{fig:study1Results}
\end{figure}

\begin{figure}[H]
\centering
\includegraphics[width=.5\textwidth]{Study1ResultsCont.png}
\caption{Study one results cont.}
\label{fig:study1ResultsCont}
\end{figure}

Total error for study one : $.09$ 


\subsection{Study Two}
In the second study, African Green Monkey Kidney Cells  Cell strain Vero E6 were again infected with Vaccinia Virus Western Reserve strain. Six sample images were captured and sent to the program for processing. Each plate contained 24 wells, and a total of 144 data points were compared with corresponding truth data points.

\begin{figure}[H]
\centering
\includegraphics[width=.5\textwidth]{Study2Results.png}
\caption{Study 2 results}
\label{fig:study2Results}
\end{figure}

\begin{figure}[H]
\centering
\includegraphics[width=.5\textwidth]{Study2ResultsCont.png}
\caption{Study 2 results cont.}
\label{fig:study2ResultsCont}
\end{figure}
Total error for study two : $.20$ 

\subsection{Study Three}
In the third study, African Green Monkey Kidney Cells  Cell strain Vero E6 were again infected with Monkey Pox Zaire strain. Three sample images were captured and sent to the program for processing. Each plate contained 18 wells, and a total of 52 data points were compared with corresponding truth data points.
\begin{figure}[H]
\centering
\includegraphics[width=.5\textwidth]{Study3Results.png}
\caption{Study three results}
\label{fig:study3Results}
\end{figure}
Total error for study three: $.20$ 


\section{Discussion}\label{sec:discussion}
Experimental data shows that the proposed method automatically counts an entire viral plaque plate with an accuracy range of 70 to 90 percent. It was observed that the accuracy of the method roughly correlates to the quality of the image acquired by the scanner and the clarity of the structures in the actual samples. Highly errant counts were tallied when the scanned image contained objects such as reflections or attributes such as low contrast. High error levels were also observed when the actual samples were smudged or otherwise damaged due to human error in preparation. In its current state, the overall accuracy of the procedure may be suitable for applications where average error levels of up to 30 percent can be tolerated. However, further investigation is needed to improve average error levels. It is speculated that methods for improving the signal to noise ratio at image acquisition stage could vastly improve overall accuracy. Use of a high resolution overhead camera and reflection reduction system such as in ~\cite{watershed} could help in this area. An overhead camera type apparatus might also prove very practical, as it could used to capture images in real time and improve overall work flow. Ultimately, a trained eye is still required to determine if a given sample plate is a candidate for automated counting; some samples are simply too damaged or noisy for the proposed method to process and produce a counts of reasonable accuracy.  


\section{Conclusion}\label{sec:conclusion}


\bibliography{IEEEabrv,mybibfile}


\section*{Acknowledgements}
Thanks to the Southern Research Institute of Frederick Maryland and Mr. Michael Inskeep for providing example plaque images and truth data.


\end{document}

